\section{漏极电流计算}

\subsection{漏极电流的核心公式推导}
漂移电流反映了载流子在电场作用下的运动
\begin{Equation}
    I=\mu Q E
\end{Equation}
扩散电流反映了载流子在梯度作用下的运动
\begin{Equation}
    I=D\dv{Q}{x}=\mu V_{th}\dv{Q}{x}
\end{Equation}
这里,$E$是电场,$Q$代表$x$处的线电荷密度,$D$在该式中代表扩散系数(而不是态密度),根据爱因斯坦关系式$D=\mu V_{th}$,扩散系数可以转化为迁移率,使漂移与扩散均能以迁移率表征。

在HEMT中,漂移电流和扩散电流均应当被考虑,有
\begin{Equation}
    I_d=-\mu_0 Q_{ch}\dv{\psi}{x}+\mu_0 V_{th}\dv{Q_{ch}}{x}
\end{Equation}
该处$\mu_0$为\cx{GaN} HEMT的低场迁移率,取$\mu_0=\SI{0.17}{m^2.V^{-1}.s^{-1}}$。

该式中$Q_{ch}=q n_s$即沿沟道方向单位长度的电荷量,参见\xref{def:Qch},而电场$E=-\dv*{\psi}{x}$则以电势负梯度的形式出现,当然,我们知道$\psi$是表面势而非电势,但由于$\psi=E_f+V_x$中的费米能级$E_f$并不随$x$变化,因此表面势梯度$-\dv*{\psi}{x}$实质上等同于电势梯度$-\dv*{V_x}{x}$。

依据\xref{def:Qch}和\xref{fml:电子气浓度与表面势}实现$Q_{ch}=C_g(V_{go}-\psi)$的代换
\begin{Equation}
    I_d=-\mu_0 W C_g(V_{go}-\psi)\dv{\psi}{x}-\mu_0 W C_g V_{th}\dv{\psi}{x}
\end{Equation}
这样一来就将变量完全转换到$\psi$上了,稍作整理
\begin{Equation}
    I_d=-\mu_0 W C_g(V_{go}+V_{th}-\psi)\dv{\psi}{x}
\end{Equation}
两边同乘$\dx$,准备积分
\begin{Equation}
    I_d\dx=-\mu_0 W C_g(V_{go}+V_{th}-\psi)\dd{\psi}
\end{Equation}
积分,右侧积分限是$\psi_s$至$\psi_d$,左侧积分限是$0$至$L$,请注意$I_d$是无关$x$的常数
\begin{Equation}
    \Int[0][L]I_d\dx=\Int[\psi_s][\psi_d]-\mu_0 W C_g(V_{go}+V_{th}-\psi)\dd{\psi}
\end{Equation}
积分结果是
\begin{Equation}
    I_d L=\mu_0 W C_g\eval{\qty[\frac{1}{2}\psi^2-(V_{go}+V_{th})\psi]}_{\psi_s}^{\psi_d}
\end{Equation}
即
\begin{Equation}
    I_d L=\mu_0 W C_g\qty[\frac{1}{2}(\psi_d^2-\psi_s^2)-(V_{go}+V_{th})(\psi_d-\psi_s)]
\end{Equation}
考虑到$\psi_d^2-\psi_s^2=(\psi_d+\psi_s)(\psi_d-\psi_s)$
\begin{Equation}
    I_d L=\mu_0 W C_g\qty[\frac{1}{2}(\psi_d+\psi_s)(\psi_d-\psi_s)-(V_{go}+V_{th})(\psi_d-\psi_s)]
\end{Equation}
将$L$移至右侧,引用\xref{def:psim}即$\psi_m=(\psi_d+\psi_s)/2$,简记$\psi_{ds}=\psi_d-\psi_s$并将其提出
\begin{Equation}
    I_d=\mu_0 C_g \frac{W}{L}\qty(\psi_m-V_{go}-V_{th})\psi_{ds}
\end{Equation}
应指出的是,上述计算出的电流是沿$x$正方向,即源至漏,而$I_d$通常是漏至源,故还应取负。
\begin{BoxFormula}[漏极电流]
    漏极电流$I_d$可以表示为
    \begin{Equation}
        I_d=\mu_0 C_g \frac{W}{L}\qty(V_{go}+V_{th}-\psi_m)\psi_{ds}
    \end{Equation}
\end{BoxFormula}

\xref{fig:漏极电流的核心公式}对$I_d$进行了可视化,取$W=L=\SI{1}{um}$,我们注意到
\begin{itemize}
    \item $I_d$的数量级在$\si{mA}$级,$I_d$随$V_d$和$V_g$的变化规律与经典的MOSFET模型是相似的。
    \item $I_d$在$V_g$较小时存在极大的反向电流,这是不正常的,存疑。
    \item $I_d$随着$V_d$的增大而增大,但是当$V_d$较大时反而减小,这是尚未考虑饱和所致。
\end{itemize}
应指出的是,对于晶体管,饱和有两种形式
\begin{itemize}
    \item 夹断饱和:随着$V_d$的增大,载流子浓度在漏端减小至零而使沟道夹断,电流饱和。
    \item 速度饱和:随着$V_d$的增大,载流子速度受到散射效应的限制达到饱和,电流饱和。
\end{itemize}
而HEMT器件中,速度饱和是饱和的主要形式(稍后将会讨论其),夹断饱和并不会发生。
\begin{Figure}[漏极电流的核心公式]
    \begin{FigureSub}[转移特性;转移特性漏极电流]
        \includegraphics[scale=0.8]{build/Chapter02C_01a.fig.pdf}
    \end{FigureSub}
    \begin{FigureSub}[输出特性;输出特性漏极电流]
        \includegraphics[scale=0.8]{build/Chapter02C_01b.fig.pdf}
    \end{FigureSub}\\ \vspace{0.25cm}
    \begin{FigureSub}[三维图;三维图漏极电流]
        \includegraphics[scale=0.8]{build/Chapter02C_01c.fig.pdf}
    \end{FigureSub}
\end{Figure}

\subsection{迁移率退化}
迁移率退化(Mobility Degradation)反映的是栅压$V_g$导致的纵向电场对迁移率的影响。
\begin{BoxFormula}[迁移率退化]
    迁移率退化通过对迁移率修正建模
    \begin{Equation}
        \mu_{eff}=\frac{\mu_0}{1+\mu_a E_{y,eff}+\mu_bE_{y,eff}^2}
    \end{Equation}
\end{BoxFormula}
其中$E_{y,eff}$是纵向有效电场,定义如下
\begin{BoxDefinition}[$E_{y,eff}$][Eyeff]
    $E_{y,eff}$定义为\footnote[2]{$E_{y,eff}$定义的实质是$E_{y,eff}=|Q_{ch}/\epsilon|$,且$Q_{ch}=C_{g}(V_{go}-\psi)$中取$\psi=\psi_m$。}
    \begin{Equation}
        E_{y,eff}=\frac{C_g}{\epsilon}|V_{go}-\psi_m|
    \end{Equation}
\end{BoxDefinition}

其中$\mu_a$和$\mu_b$分别是迁移率退化的系数,取$\mu_a=\SI{e-9}{V^{-1}}$和$\mu_b=\SI{e-18}{V^{-1}}$。

\xref{fig:迁移率受迁移率退化的影响}展示了考虑迁移率退化的迁移率$\mu_{eff}$随$V_g,V_d$的变化,观察到
\begin{itemize}
    \item 注意!此处的$z$轴是上下翻转的,下方迁移率较大,上方迁移率较小。
    \item $\mu_{eff}$随$V_{g}$的增大而增大,当$V_{g}>V_{off}$时随$V_{d}$的增大而减小。
    \item $\mu_{eff}$曲面中的折痕是其表达式中的绝对值所致。
    \item $\mu_{eff}$相较低场迁移率$\mu_0$并未减小太多,这表明迁移率降低的影响并不显著。
\end{itemize}


\begin{Figure}[迁移率受迁移率退化的影响]
    \includegraphics[scale=0.8]{build/Chapter02C_01i.fig.pdf}
\end{Figure}

\xref{fig:漏极电流--迁移率退化}对比了迁移率取$\mu_0,\mu_{eff}$下的电流曲线,观察到
\begin{itemize}
    \item 迁移率降低下$\mu_{eff}$相较$\mu_0$并未改变太多,故电流曲线也只是稍有偏移。
    \item 迁移率降低会使$I_d$比原先略小一些。
\end{itemize}
\begin{Figure}[漏极电流--迁移率退化]
    \begin{FigureSub}[转移特性;转移特性迁移率退化]
        \includegraphics[scale=0.8]{build/Chapter02C_01d.fig.pdf}
    \end{FigureSub}
    \begin{FigureSub}[输出特性;输出特性迁移率退化]
        \includegraphics[scale=0.8]{build/Chapter02C_01e.fig.pdf}
    \end{FigureSub}
\end{Figure}

\subsection{速度饱和效应}
速度饱和效应(Velocity Saturation)是指,载流子的速度应当随着$V_{d}$的增大而增大,但实际上,载流子的速度并不可能无限增加。当载流子的速度很快时,由于晶格的散射作用,载流子的速度$v$趋于定值$v_{sat}$,这就是速度饱和效应。速度饱和同样可以通过迁移率的降低来建模。
\begin{BoxFormula}[速度饱和效应]
    速度饱和效应通过对迁移率修正建模
    \begin{Equation}
        \mu_{sat}=\frac{\mu_{eff}}{\sqrt{1+(E_x\mu_{eff}/v_{sat})^2}}
    \end{Equation}
\end{BoxFormula}

其中$E_x$是横向电场,定义如下
\begin{BoxDefinition}[$E_x$][Ex]
    $E_x$定义为
    \begin{Equation}
        E_x=\frac{\psi_d-\psi_s}{L}
    \end{Equation}
\end{BoxDefinition}
其中$v_{sat}$是饱和速度,取$v_{sat}=\SI{1.9e5}{m.s^{-1}}$。

\xref{fig:迁移率受速度饱和效应的影响}展示了考虑速度饱和效应的迁移率$\mu_{sat}$随$V_g,V_d$的变化,观察到
\begin{itemize}
    \item $\mu_{sat}$主要受$V_d$的影响,$\mu_{sat}$随着$V_d$的增大迅速减小并接近零。
    \item $\mu_{sat}$会显著小于$\mu_0$,这表明,速度饱和效应对迁移率的影响非常大。
\end{itemize}
\begin{Figure}[迁移率受速度饱和效应的影响]
    \includegraphics[scale=0.8]{build/Chapter02C_01j.fig.pdf}
\end{Figure}

\xref{fig:漏极电流--速度饱和效应}展示了速度饱和效应下的电流曲线,观察到
\begin{itemize}
    \item 请注意,\xref{fig:漏极电流--速度饱和效应}的纵坐标尺度相较前续图像发生了变化。
    \item $I_d$的典型值在考虑速度饱和后,从原先的十几$\si{mA}$减小到了几$\si{mA}$。
    \item $I_d$仍然会随$V_d$的增大而减小,但从原先近似抛物型的下降变为近似线性的下降。
\end{itemize}
\begin{Figure}[漏极电流--速度饱和效应]
    \begin{FigureSub}[转移特性;转移特性速度饱和]
        \includegraphics[scale=0.8]{build/Chapter02C_01f.fig.pdf}
    \end{FigureSub}
    \begin{FigureSub}[输出特性;输出特性速度饱和]
        \includegraphics[scale=0.8]{build/Chapter02C_01g.fig.pdf}
    \end{FigureSub}\\ \vspace{0.25cm}
    \begin{FigureSub}[三维图;三维图速度饱和]
        \includegraphics[scale=0.8]{build/Chapter02C_01h.fig.pdf}
    \end{FigureSub}
\end{Figure}

\subsection{沟道调制效应}
沟道调制效应(Channel Length Modulation, CLM)反映为对$I_d$的修正。

\begin{BoxFormula}[沟道调制效应]
    沟道调制效应通过对$I_d$的修正建模
    \begin{Equation}
        I_{d,CLM}=I_d[1+\lambda(V_{dsx}-V_{ds,eff})]
    \end{Equation}
\end{BoxFormula}

这里$\lambda$是沟道调制效应的系数,默认值为$\lambda=\SI{0.0}{V^{-1}}$,这里为考察其影响取$\lambda=\SI{0.1}{V^{-1}}$。

其中$V_{dsx}$是一个相当接近$V_{ds}$的表达式
\begin{BoxDefinition}[$V_{dsx}$][Vdsx]
    $V_{dsx}$定义为
    \begin{Equation}
        V_{dsx}=\sqrt{V_{ds}^2+0.01}
    \end{Equation}
\end{BoxDefinition}\goodbreak
其中$V_{ds,eff}$是一个与$V_{ds}$相关的表达式\nopagebreak
\begin{BoxDefinition}[$V_{ds,eff}$][Vdseff]
    $V_{ds,eff}$定义为
    \begin{Equation}
        V_{ds,eff}=V_{ds}\qty[1+\qty(\frac{V_{ds}}{V_{ds,sat}})^{\delta}]^{-1/\delta}
    \end{Equation}
\end{BoxDefinition}

其中$V_{ds,sat}$是一个由$V_{go,eff}$决定的表达式,$V_{go,eff}$参见\xref{def:Vgoeff}
\begin{BoxDefinition}[$V_{ds,sat}$][Vdssat]
    $V_{ds,sat}$定义为
    \begin{Equation}
        V_{ds,sat}=\frac{(2v_{sat}/\mu_{eff})L\cdot V_{go,eff}}{(2v_{sat}/\mu_{eff})L+V_{go,eff}}
    \end{Equation}
\end{BoxDefinition}

这里$\delta$是$V_{ds,eff}$中的指数,取$\delta=2$。

\begin{Figure}[漏极电流--沟道调制效应]
    \begin{FigureSub}[转移特性;转移特性沟道调制]
        \includegraphics[scale=0.8]{build/Chapter02C_01k.fig.pdf}
    \end{FigureSub}
    \begin{FigureSub}[输出特性;输出特性沟道调制]
        \includegraphics[scale=0.8]{build/Chapter02C_01l.fig.pdf}
    \end{FigureSub}
\end{Figure}

\xref{fig:漏极电流--沟道调制效应}展示了沟道调制对$I_d$的影响,输出特性中$I_d$会因该效应略微上翘些。

\subsection{漏致势垒降低}
漏致势垒降低(Drain Induced Barrier Lowering, DIBL)反映为对$V_{off}$的修正。
\begin{BoxFormula}[漏至势垒降低]
    漏致势垒降低通过对$V_{off}$的修正建模
    \begin{Equation}
        V_{off,DIBL}=V_{off}-\eta_0\frac{V_{dsx}V_{ds,scale}}{\sqrt{V_{dsx}^2+V_{ds,scale}^2}}
    \end{Equation}
\end{BoxFormula}
这里$\eta_0$和$V_{ds,scale}$均为DIBL相关系数,取$\eta_0=\SI{e-3}{}$和$V_{ds,scale}=\SI{5}{V}$。

\begin{Figure}[漏致势垒降低]
    \includegraphics[scale=0.8]{build/Chapter02C_01m.fig.pdf}
\end{Figure}
\xref{fig:漏致势垒降低}展示了漏致势垒降低对$V_{off}$的影响,其使$V_{off}$随$V_d$减小,但影响非常轻微。