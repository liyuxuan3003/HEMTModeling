\section{通道电阻和接触电阻计算}
实际上,源极和漏极与沟道之间都存在相当的电阻,这由两部分组成
\begin{itemize}
    \item 通道电阻(Access Region Resistance):指源或漏与沟道之间区域的电阻。这里需要解释一下,在HEMT中,栅的边界并不会紧贴源和漏,相反,栅与源和漏之间有相当一段距离,这段距离被称为通道区(Access Region),通道电阻相当重要且具有非线性特性。
    \item 接触电阻(Contact Resistance):指源或漏与其金属间的金半接触电阻。
\end{itemize}

以下我们均先以源端的通道电阻和接触电阻为例计算,漏端的情形与源端完全一致。

\subsection{通道电阻}
通道电阻$R_{acc,s}$最基本的一个表达式是(这里下标$s$代表源端,而$acc$代表通道)
\begin{Equation}
    R_{acc,s}=\frac{L_{SG}}{W\cdot N_F\cdot q\cdot n_{s,acc,s}\cdot \mu_{acc,s}}
\end{Equation}
这里做一些解释:首先,我们或许还记得半导体物理告诉我们电导率$\sigma= qn\mu$,这是电阻表达式的核心部分,这里,$n_{s,acc,s}$是通道处的电子气浓度,$\mu_{acc,s}$是通道处的电子迁移率,它们都是常数,取$n_{s0,acc,s}=\SI{1e17}{m^{-2}}$和$\mu_{acc,s}=\SI{0.155}{m^2.V^{-1}.s^{-1}}$,沟道下这两个物理量的情况参见\xref{fig:迁移率受速度饱和效应的影响}和\xref{fig:电子气浓度}。随后,从电导率到电阻,显然还需要一些长宽的系数,$L_{SG}$是源至栅的距离,也就是通道区的长度,$W$是晶体管宽度,取$L_{SG}=\SI{1}{um}$,作为参照,晶体管的宽和长也是取$W=\SI{1}{um}$和$L=\SI{1}{um}$。另外,$N_F$代表Number of Fingers,$N_F$是与版图结构有关的整数,默认取$N_F=1$,简单来说,这里可以认为$W\cdot N_F$是代表了等效的宽度。\footnote{原文$n_{s,acc,s}=\SI{5e17}{m^{-2}}$,该值任老师认为太高,改为$n_{s,acc,s}=\SI{1e17}{m^{-2}}$。}\goodbreak

但仅这样是不够的,由于沟道电流$I_d$与通道区电流$I_{acc,s}$应相等(并不清楚$I_{acc,s}$的表达式以及$I_{acc,s}=I_d$的影响,这只是原文对以下修正的说明),通道电阻应修正为
\begin{Equation}
    R_{acc,s}=\frac{L_{SG}}{W\cdot N_F\cdot q\cdot n_{s,acc,s}\cdot \mu_{acc,s}}\qty[1-\qty(\frac{I_d}{I_{sat,acc,s}})^{m_{acc,s}}]^{-1/m_{acc,s}}
\end{Equation}
这里$m_{acc,s}$是一个指数项,取$m_{acc,s}=2$,而$I_{sat,acc,s}$是通道电流$I_{acc,s}$的最大值
\begin{Equation}
    I_{sat,acc,s}=W\cdot N_F\cdot n_{s,acc,s}\cdot v_{sat,acc,s}
\end{Equation}
这里$v_{sat,acc,s}$是通道区载流子的饱和速度,取$\SI{2e5}{m.s^{-1}}$。\footnote{原文$v_{sat,acc,s}=\SI{0.5e5}{m.s^{-1}}$,该值任老师认为太高,改为$v_{sat,acc,s}=\SI{2e5}{m.s^{-1}}$。}

\begin{BoxFormula}[源侧通道电阻]
    源侧通道电阻$R_{acc,s}$可以表示为
    \begin{Equation}
        \qquad\qquad
        R_{acc,s}=\frac{L_{SG}}{W\cdot N_F\cdot q\cdot n_{s,acc,s}\cdot \mu_{acc,s}}\qty[1-\qty(\frac{I_d}{I_{sat,acc,s}})^{m_{acc,s}}]^{-1/m_{acc,s}}
        \qquad\qquad
    \end{Equation}
    源侧通道区饱和电流$I_{sat,acc,s}$可以表示为
    \begin{Equation}
        I_{sat,acc,s}=W\cdot N_F\cdot n_{s,acc,s}\cdot v_{sat,acc,s}
    \end{Equation}
\end{BoxFormula}
对应的写出漏端的公式
\begin{BoxFormula}[漏侧通道电阻]
    漏侧通道电阻$R_{acc,d}$可以表示为
    \begin{Equation}
        \qquad\qquad
        R_{acc,d}=\frac{L_{DG}}{W\cdot N_F\cdot q\cdot n_{s,acc,d}\cdot \mu_{acc,d}}\qty[1-\qty(\frac{I_d}{I_{sat,acc,d}})^{m_{acc,d}}]^{-1/m_{acc,d}}
        \qquad\qquad
    \end{Equation}
    漏侧通道区饱和电流$I_{sat,acc,d}$可以表示为
    \begin{Equation}
        I_{sat,acc,d}=W\cdot N_F\cdot n_{s,acc,d}\cdot v_{sat,acc,d}
    \end{Equation}
\end{BoxFormula}

\subsection{接触电阻}
接触电阻相较通道电阻简单很多,是一个定值。
\begin{BoxFormula}[源侧接触电阻]
    源侧接触电阻$R_{contact,s}$可以表示为
    \begin{Equation}
        R_{contact,s}=\frac{R_{sc}}{W\cdot N_F}
    \end{Equation}
\end{BoxFormula}
这里$R_{sc}$取$R_{sc}=\SI{1e-4}{\ohm.m^{-1}}$,类似的,也可以写出漏极侧的公式。
\begin{BoxFormula}[漏侧接触电阻]
    漏侧接触电阻$R_{contact,d}$可以表示为
    \begin{Equation}
        R_{contact,d}=\frac{R_{dc}}{W\cdot N_F}
    \end{Equation}
\end{BoxFormula}

\subsection{可视化分析}

\xref{fig:通道电阻和接触电阻}展示了通道电阻和接触电阻随$I_d$的变化
\begin{itemize}
    \item 接触电阻$R_{acc}$在$I_d=0$时约为$\SI{400}{\ohm}$,其随$I_d$的增大而增大,并最终在$I_d=I_{sat,acc}$时(约为$\SI{3.2}{mA}$)趋于无穷大。这是合理的,通道电阻确实会限制了HEMT的最大电流,而$\SI{3}{mA}$左右的最大电流对于一个宽和长均为$\SI{1}{um}$的HEMT器件而言是合理的。
    \item 通道电阻$R_{contact}$是一个定值,其约为$\SI{100}{\ohm}$。
\end{itemize}
\begin{Figure}[通道电阻和接触电阻]
    \includegraphics[scale=0.8]{build/Chapter02D_01a.fig.pdf}
\end{Figure}